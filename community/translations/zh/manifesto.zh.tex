% Chinese translation of the OCRC manifesto
% To be completed by native speakers

\documentclass[12pt,a4paper]{article}
\usepackage[UTF8]{ctex}
\usepackage{geometry}
\geometry{top=2cm,bottom=2cm,left=2.5cm,right=2.5cm}
\usepackage{hyperref}
\hypersetup{colorlinks=true, linkcolor=blue, urlcolor=blue}

\title{开放合作研究协议 \\ 跨学科对话与知识可持续发展的框架标准 \\ (Open Cooperative Research Protocol --- OCRC)}
\author{由快速射电暴本质与开放科学的跨学科对话参与者提出}
\date{版本 1.0 --- 2026年3月}

\begin{document}

\maketitle

\begin{abstract}
本文描述了一个自愿性框架协议,旨在组织开放、合作、跨学科的科学活动。该协议不引入强制性标准,而是提供一套商定原则、句法和语义结构,以促进数据交换、共同问题设定和知识的可持续发展。主要组成部分包括:统一的研究对象标识符(ROI)、RO-Crate数据打包格式、OCRC核心本体、假设描述框架(HDF),以及生态认知和合作互动的伦理原则。该协议面向从天体物理学到生物经济学等广泛学科,旨在为知识共同空间的自然生长创造条件,无需集中强制。
\end{abstract}

\tableofcontents
\newpage

\section{前言}
(此处为 manifesto 中文翻译内容,待补充)

% ... 完整翻译需基于英文/俄文原稿

\end{document}
