% Русская версия манифеста OCRC
% (Полный исходник уже есть в docs/manifesto.tex, здесь можно разместить копию или ссылку)

\documentclass[12pt,a4paper]{article}
\usepackage[utf8]{inputenc}
\usepackage[T2A]{fontenc}
\usepackage[russian]{babel}
\usepackage{amsmath,amsfonts,amssymb}
\usepackage{graphicx}
\usepackage{hyperref}
\usepackage{xcolor}
\usepackage{listings}
\usepackage{booktabs}
\usepackage{longtable}
\usepackage{geometry}
\geometry{top=2cm,bottom=2cm,left=2.5cm,right=2.5cm}

\title{Открытый протокол кооперативных исследований: \\ Рамочный стандарт для междисциплинарного диалога и устойчивого развития знаний \\ (OCRC)}
\author{Предложено участниками междисциплинарного диалога о природе быстрых радиовсплесков и открытой науке}
\date{Версия 1.0 --- март 2026}

\begin{document}

\maketitle

\begin{abstract}
Настоящий документ описывает добровольный рамочный протокол для организации открытой, кооперативной и междисциплинарной научной деятельности. Протокол не вводит обязательных стандартов, а предлагает набор согласованных принципов, синтаксических и семантических структур, облегчающих обмен данными, совместную постановку задач и устойчивое развитие знаний. Основные компоненты включают: унифицированные идентификаторы исследовательских объектов (ROI), формат упаковки данных RO-Crate, базовую онтологию OCRC Core, язык описания гипотез HDF, а также этические принципы экологичного познания и кооперативного взаимодействия. Протокол ориентирован на широкий круг дисциплин --- от астрофизики до биоэкономики --- и призван создать условия для естественного роста общего пространства знаний без централизованного принуждения.
\end{abstract}

\tableofcontents
\newpage

\section{Преамбула}
(Полный текст манифеста, идентичный docs/manifesto.tex)

% ... остальное содержимое

\end{document}
