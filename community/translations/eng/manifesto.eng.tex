% English translation of the OCRC manifesto
% (Based on the original Russian/English version)

\documentclass[12pt,a4paper]{article}
\usepackage[utf8]{inputenc}
\usepackage[T1]{fontenc}
\usepackage{amsmath,amsfonts,amssymb}
\usepackage{graphicx}
\usepackage{hyperref}
\usepackage{xcolor}
\usepackage{listings}
\usepackage{booktabs}
\usepackage{longtable}
\usepackage{geometry}
\geometry{top=2cm,bottom=2cm,left=2.5cm,right=2.5cm}

\title{Open Cooperative Research Protocol: \\ A Framework Standard for Interdisciplinary Dialogue and Sustainable Knowledge Development \\ (OCRC)}
\author{Proposed by participants in the interdisciplinary dialogue on the nature of fast radio bursts and open science}
\date{Version 1.0 --- March 2026}

\begin{document}

\maketitle

\begin{abstract}
This document describes a voluntary framework protocol for organizing open, cooperative, and interdisciplinary scientific activity. The protocol does not impose mandatory standards but offers a set of agreed principles, syntactic and semantic structures that facilitate data exchange, joint problem formulation, and sustainable knowledge development. Main components include: unified Research Object Identifiers (ROI), the RO-Crate data packaging format, the OCRC Core Ontology, the Hypothesis Description Framework (HDF), and ethical principles of ecological cognition and cooperative interaction. The protocol is oriented towards a wide range of disciplines — from astrophysics to bioeconomics — and aims to create conditions for the natural growth of a common knowledge space without centralized enforcement.
\end{abstract}

\tableofcontents
\newpage

\section{Preamble}
(Full English translation of the manifesto would go here.)

% ... rest of the document based on the original LaTeX source

\end{document}
